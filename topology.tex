\documentclass{article}
\usepackage{geometry}
\usepackage{graphicx} % Required for inserting images
\usepackage{amsmath, amsthm, amssymb}
\usepackage{parskip}
\newgeometry{vmargin={15mm}, hmargin={24mm,34mm}}
\theoremstyle{definition} 
\newtheorem{definition}{Definition}

\newtheorem{theorem}{Theorem}[section]
\newtheorem{lemma}[theorem]{Lemma}
\newtheorem{corollary}{Corollary}[theorem]
\newtheorem{example}{Example}[theorem]

\newcommand{\N}{\mathbb{N}}
\newcommand{\Z}{\mathbb{Z}}
\newcommand{\R}{\mathbb{R}}
\newcommand{\Q}{\mathbb{Q}}
\newcommand{\fdiff}{f^{\prime}}
\newcommand{\interior}{\text{int}}
\newcommand{\dom}{\text{Dom}}
\newcommand{\ran}{\text{Ran}}

\title{Topology Notes}
\date{March 2024}

\begin{document}

\maketitle

\section{Definitions and Motivation}

\begin{definition}
    Let $X$ be a set. $\tau \subseteq \mathcal{P}(X)$ is said to be a \textbf{topology on $X$} if

    \begin{enumerate}
        \item $\varnothing, X \in \tau$
        \item $\tau$ is closed under unions.
        \item $\tau$ is closed under finite intersections. 
    \end{enumerate}
\end{definition}

Equivalently, we can define a topology using its closed sets. See Exercise 17.1 in
Monkres Topology.




\begin{example}
    A topology where the only open sets are $\varnothing$ and $X$ is called the \textbf{trivial (indiscrete) topology}.
\end{example}

\begin{example}
    Let $X$ be a set. Let $\tau$ be the set of all cofinite subsets of $X$.
    $(X,\tau)$ is called the \textbf{cofinite (finite complement)} topology
    of $X$.
\end{example}

Similar to the example above, we can define an open set as having a countable
complement and get a topology on $X$.

\begin{definition}
    Given an ordered set $X$, a subset $Y$ of $X$ is said to be \textbf{convex}
    if for all $a < b$ in $Y$, $(a,b) \subseteq Y$.
\end{definition}

Intervals and rays are convex in $X$.

\newpage

\section{Basis of a Topology}

\begin{definition}
    Let $X$ be a set. A basis $\mathcal{B}$ for $(X,\tau)$ is a collection of subsets
    of $X$ such that

    \begin{enumerate}
        \item $\forall x \in X: \exists B \in \mathcal{B}: x \in B$
        \item $\forall x \in X: \forall B_{1},B_{2} \in \mathcal{B}: x \in B_{1} \cap B_{2} \implies
        (\exists B_{3} \in \mathcal{B}: x \in B_{3} \land B_{1} \cap B_{2} \subseteq B_{3})$
    \end{enumerate}
\end{definition}

Given a basis $\mathcal{B}$, we can generate a topology of $\tau$ by defining the open sets
as in the following lemma:

\begin{lemma}
    $A \in X$ is open if and only if $\forall a \in A: \exists U \in \tau: a \in U \land U \subseteq A$.

    In other words, $A$ is open if and only if every point in $A$ is contained in an open ball
    that's contained in $A$.
\end{lemma}
\begin{proof}
    The forward implication is trivial, just pick $U = A$. Let's now prove the reverse implication.
    Assume a set $A$ satisfies the condition on the right-hand side. Let $U_{a}$ be the open set
    containing $a$. Notice that $A = \bigcup_{a \in A} U_{a}$, so $A$ is a union of open sets and
    therefore open.
\end{proof}

\newpage

\section{Subspaces}

\begin{lemma}
    Let $X$ be an ordered set with the order topology and $Y$ be a convex subset of $X$.
    Then, the order topology on $Y$ is the same topology as the one $Y$ inherits as a subspace
    of $X$.
\end{lemma}




\newpage

\section{Continuity}

The continuity of a function $f: X \xrightarrow{} Y$ doesn't depend only on the function
but on the topologies put on $X$ and $Y$.

\begin{definition}
    A function $f: X \xrightarrow{} Y$ is \textbf{continuous} if it preimages
    open sets of $Y$ into open sets of $X$. More formally, 
\end{definition}

If $\mathcal{B}$ is a basis for $Y$, it suffices to show that $f$ preimages
every $B \in \mathcal{B}$ to an open set of $X$.



\newpage

\section{Homeomorphisms}

\begin{definition}
    Let $X,Y$ be topological spaces and $f: X \xrightarrow{} Y$ be a homeomorphism.
\end{definition}

\begin{definition}
    Let $X,Y$ be topological spaces $f: X \xrightarrow{} Y$ be an
    injective continuous function. If $f^{\prime}: X \xrightarrow{} f(X)$
    is an homeomorphism, $f$ is called a \textbf{topological imbedding}.
\end{definition}

\begin{lemma}
    Let $X,Y$ be metric spaces and $f: X \xrightarrow{} Y$ be an isometry.
    Then, $f$ is an imbedding.
\end{lemma}
\begin{proof}
    $f$ is (uniformly) continuous and injective. Clearly, every restriction of
    $f$ is also continuous.
\end{proof}

\begin{lemma}
    Let $f: X \xrightarrow{} Y$ be a homeomorphism and $Z$ be a subspace of $X$.
    Then, $f$ restricred to $Z$ is still an homeomorphism.
\end{lemma}

\begin{lemma}
    $[a,b] \simeq (c,d)$
\end{lemma}
\begin{proof}
    Assume by contradiction that there exists a homeomorphism 
    $f: [a,b] \xrightarrow{} (c,d)$.


\end{proof}

\newpage

\section{Closed Sets}


\begin{lemma}
    If $A$ is closed in $Y$ and $Y$ is closed in $X$,
    $A$ is closed in $X$.
\end{lemma}


\newpage

\section{Connectedness}

\begin{definition}
    Let $(X,\tau)$ be a topological space. $X$ is said to be \textbf{disconnected}
    if there are disjoint open sets $U,V$ such $U \cup V = X$.
\end{definition}

\begin{definition}
    $X$ is said to be \textbf{connected} if it's not disconnected.
\end{definition}

\begin{lemma}
    $X$ is connected if and only if the only clopen sets in $X$ are $\varnothing$ and $X$.
\end{lemma}

\begin{definition}
    $X$ is said to be separated to disjoint sets $A,B$ if $A,B$ don't contain each other's limit points
    and $A \cap B = X$.
\end{definition}

\begin{lemma}
    $X$ is connected if and only if it doesn't separate into two sets $A,B$.
\end{lemma}

\begin{lemma}
    Let $C$ and $D$ be a separation of $X$ and $Y$ be a connected subspace of $X$.
    Then, $Y$ lies entirely in $C$ or $D$.
\end{lemma}
\begin{proof}
    Notice that $C \cap Y$ and $D \cap Y$ are disjoint open sets in $Y$ and notice
    that their union is $Y$.
\end{proof}

\begin{lemma}
    The union of a collection of subspaces $A_{\alpha}$ that have a point
    in common is connected.
\end{lemma}



\subsection{Exercises}

Let $\tau, \tau^{\prime}$ be two topologies on $X$. If $\tau$ is not connected, $\tau^{\prime}$ is not connected. 
If $\tau^{\prime}$ is connected, $\tau$ is also connected.

\begin{lemma}
    Let $A_{n}$ be a sequence of connected subspaces of $X$. Assume that
    $A_{n} \cap A_{n+1} \neq \varnothing$. Prove that $A = \bigcup_{n = 1}^{\infty} A_{n}$
    is connected.
\end{lemma}

The intuition is the following: the common point between consecutive subspaces acts
as a bridge. I think this proof can be formalized by using contradiction and then
getting two points in two sets, then arguing they have to be connected.

\begin{proof}
    
\end{proof}

\begin{lemma}
    Let $X$ be a topological space and $A_{\alpha \in I}$ be a collection
    of connected subspaces. Let $A$ be a connected subspace with a non-trivial
    intersection with all $A_{\alpha}$. Then, $A \cup A_{\alpha \in I}$ is
    also connected.
\end{lemma}

Here, it's as if we have a single connection point connecting all the other
sets together.

\begin{proof}
    
\end{proof}

Let $X$ be an infinite set in the cofinite topology. Then, $X$ is connected since
otherwise $X$ would be finite.

\begin{definition}
    A space is \textbf{totally disconnected} if the only connected subspaces
    are one-point sets.
\end{definition}

Notice that if we equip a set with the discrete topology, it's totally disconnected.


\newpage

\section{Connected Subspaces of the Real Line}

\begin{definition}
    A simply ordered set $L$ containing more than one element is called a \textbf{linear continuum}
    if the following hold:

    \begin{enumerate}
        \item $L$ has the least upper bound property.
        \item[Archimedean Property] For any two elements $x < y$, $\exists z \in L: x < z < y$.
    \end{enumerate}
\end{definition}

\begin{theorem}
    If $L$ is a linear continuum in the order topology, every convex
    subspace of $L$ is connected.
\end{theorem}
\begin{proof}
    
\end{proof}

\begin{lemma}
    $(0,1), [0,1), [0,1]$ are not homemorphic.
\end{lemma}
\begin{proof}
    Consider a homemorphism $f: [0,1) \xrightarrow{} (0,1)$.
    Then, the restriction of this homemorphism on $(0,1)$
    is still an homemorphism. However, the image of this map is no longer
    connected.

    We can use a similar argument for the rest of the arguments.
\end{proof}

\newpage

\section{Compactness}

\begin{definition}
    Let $X$ be a topological space. $X$ is called \textbf{separable} if
    it has a countable dense subset.
\end{definition}

\newpage

\section{Convergence in Topology}

\begin{definition}
    A sequence $x_{n}$ is said to converge to $x$ if for every neighborhood
    $U$ of $x$, there's some $N \in \N: \forall n \geq N: x_{n} \in U$.
\end{definition}

\newpage

\section{Separation Axioms}

The separation axioms try to mimic the properties of metric spaces.

\begin{definition}
    A topological space is said to be $T_{0}$ if any two distinct points in $X$ are topologically distinguisable.
\end{definition}

\begin{definition}
    A topological space $X$ satisfies $T_{1}$ if for any $x,y \in X$ we can find 
    a neighborhood $U$ of $x$ such that $y \notin U$.
\end{definition}

\begin{lemma}
    A topological space $X$ satisfies $T_{1}$ if and only if every singleton set is closed.
\end{lemma}

\begin{corollary}
    If a topological space $X$ satisfies $T_{1}$, every finite point set of $X$ is closed.
\end{corollary}

\begin{theorem}
    Let $X$ be a space satisfying $T_{1}$ and $A \subseteq X$. Then, $x$ is a
    limit point of $A$ if and only if every neighborhood of $x$ contains 
    infinitely many points of $A$.
\end{theorem}
\begin{proof}
    Assume $x$ is a limit point of $A$ and assume by contradiction that there's
    some neighborhood $U$ of $x$ such that $U$ contains finitely many points of $A$.
\end{proof}

\begin{definition}
    A topological space $X$ satisfies $T_{2}$ or is \textbf{Hausdorff} if for any $x,y \in X$ we can find disjoint open
    sets $U,V$ such that $x \in U$ and $y \in V$.
\end{definition}

\begin{lemma}
    Sequential limits in Hausdorff spaces are unique.
\end{lemma}
\begin{proof}
    Let $x_{n}$ be a sequence in $X$ and assume $x_{n}$ converges to $x$. Let $y \neq x$.
    Then, there's some $U_{x}$, $U_{y}$ disjoint neighborhoods of $x$ and $y$. Since $x_{n}$
    converges to $x$, there's some $N \in \N: \forall n \geq N: x_{n} \in U_{x} \implies x_{n} \notin U_{y}$.
    Thus, $x_{n}$ doesn't converge to $y$.
\end{proof}

\begin{lemma}
    Every simply ordered set is a Hausdorff space in the order topology.
\end{lemma}

\begin{lemma}
    Hausdorff spaces are closed under products.
\end{lemma}

\begin{lemma}
    Subspaces of Hausdorff spaces are Hausdorff.
\end{lemma}

\begin{lemma}
    $X$ is Hausdorff if and only if the diagonal $\Delta = \{ (x,x): x \in X \}$
    is closed in $X \times X$.
\end{lemma}
\begin{proof}
    
\end{proof}


\begin{definition}
    A topological space $X$ satisfies $T_{3}$ or is \textbf{regular} if for any point $a$ and closed set
    $B \subseteq X$ there are disjoint open sets $U,V$ such that $a \in U$ and $B \subseteq V$.
\end{definition}

\begin{example}
    The Zariski topology on $\R$ is not Hausdorff.
\end{example}

\begin{definition}
    A topological space $X$ satisfies $T_{4}$ or is \textbf{normal} if for any two closed sets
    $A,B \subseteq X$ there are disjoint open sets $U,V$ such that $A \subseteq U$ and $B \subseteq V$.
\end{definition}

\begin{lemma}
    
\end{lemma}

\begin{lemma}
    Every metric space is normal.
\end{lemma}
\begin{proof}
    
\end{proof}

\newpage

\section{Countability Axioms}

\begin{definition}
    A topological space $X$ is said to be \textbf{second-countable} if it
    has a countable basis.
\end{definition}

Not every metric space has a second-countable basis. 

\begin{theorem}
    Assume $X$ is second-countable. Then,

    \begin{enumerate}
        \item Every open covering of $X$ has a countable subcover.
        \item $X$ is separable.
    \end{enumerate}
\end{theorem}

\begin{proof}
    
\end{proof}

\newpage


\end{document}